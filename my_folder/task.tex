%%%% Начало оформления заголовка - оставить без изменений !!! %%%%
\thispagestyle{empty}%
\setcounter{tskPageFirst}{\value{page}} %сохранили номер первой страницы Задания
\ifnumequal{\value{tskPrint}}{1}{% если двухсторонняя печать Задания, то...
	\newgeometry{twoside,top=2cm,bottom=2cm,left=3cm,right=1cm,headsep=0cm,footskip=0cm}
	\savegeometry{MyTask} %save settings
	\makeatletter % задаём оформление второй страницы ВКР как нечетной, а третьей - как чётной
	\checkoddpage % проверка четности из memoir-класса
	\ifoddpage
	\else
		\let\tmp\oddsidemargin
		\let\oddsidemargin\evensidemargin
		\let\evensidemargin\tmp
		\reversemarginpar
	\fi
	\makeatother
}{} % 
\pagestyle{empty} % удаляем номер страницы на втором/третьем листе
\makeatletter
\newrefcontext[labelprefix={3.}] % задаём префикс для списка литературы
\makeatother
\setlength{\parindent}{0pt}
{\centering\bfseries%
%	\small	% настройки - начало 
	
				{%\normalfont %2020
						\MakeUppercase{\SPbPU}}\\
				\institute

\par}\intervalS% завершает input

				\noindent
				\begin{minipage}{\linewidth}
				\vspace{\mfloatsep} % интервал 	
				\begin{tabularx}{\linewidth}{Xl}
					&УТВЕРЖДАЮ      \\
					&\HeadTitle     \\			
					&\underline{\hspace*{0.1\textheight}} \Head     \\
					&<<\underline{\hspace*{0.05\textheight}}>> \underline{\hspace*{0.1\textheight}} \thesisYear г.  \\  
				\end{tabularx}
				\vspace{\mfloatsep} % интервал 	
				\end{minipage}

\intervalS{\centering\bfseries%

				ЗАДАНИЕ\\
				на выполнение %с 2020 года 
				%по выполнению % до 2020 года
				выпускной квалификационной работы


\intervalS\normalfont%

				студенту \AuthorFullDat{} гр.~\group


\par}\intervalS%
%%%%
%%%% Конец оформления заголовка  %%%%
 	
	
	
\begin{enumerate}[1.]
	\item Тема работы: {\expandafter \thesisTitle.}
	%\item Тема работы (на английском языке): \uline{\thesisTitleEn.} % вероятно после 2021 года
	\item Срок сдачи студентом законченной работы: \thesisDeadline. 
	\item Исходные данные по работе: Набор данных для обучения и тестирования, представленный парами объектов следующих типов: изображение с микроскопа разной степени четкости и вещественное число со знаком, означающее направление и расстояние до фокальной плоскости.\\%
	%\printbibliographyTask % печать списка источников % КОММЕНТИРУЕМ ЕСЛИ НЕ ИСПОЛЬЗУЕТСЯ
	% В СЛУЧАЕ, ЕСЛИ НЕ ИСПОЛЬЗУЕТСЯ МОЖНО ТАКЖЕ ЗАЙТИ В setup.tex и закомментировать \vspace{-0.28\curtextsize}
	Инструментальные средства:
	\begin{itemize}
		\item Язык программирования python
		\item Среда разработки Visual Studio Code
		\item Система контроля версий git
	\end{itemize}
	Ключевые источники литературы:
	\begin{enumerate}
		\item Ho C. J., Chan C. C., Chen H. H. AF-Net: A convolutional neural network approach to phase detection autofocus //IEEE Transactions on Image Processing. – 2020. – Т. 29. – С. 6386-6395.
%		\item Liao J. et al. Deep learning-based single-shot autofocus method for digital microscopy //Biomedical Optics Express. – 2022. – Т. 13. – №. 1. – С. 314-327.
		\item Howard A. G. et al. Mobilenets: Efficient convolutional neural networks for mobile vision applications //arXiv preprint arXiv:1704.04861. – 2017.
	\end{enumerate}
	\item Содержание работы (перечень подлежащих разработке вопросов):
	\begin{enumerate}[label=\theenumi\arabic*.]
		\item Введение. Обоснование актуальности проблемы.
		\item Постановка задачи.
		\begin{enumerate}[label=\theenumii\arabic*.]
			\item Основные уравнения, критерии качества.
		\end{enumerate}
		\item Обзор существующих решений.
		\begin{enumerate}[label=\theenumii\arabic*.]
			\item Классические метолы.
			\item Нейросетевые метолы.
		\end{enumerate}
		\item Разработка нового нейросетевого решения.
		\item Эксперименты и тесты.
		\item Результаты и сравнения с существующими решениями.
		\item Выводы
		\item Заключение 
	\end{enumerate}
	%\item Перечень графического материала (с указанием обязательных чертежей): 
	%\begin{enumerate}[label=\theenumi\arabic*.]
	%	\item Схема работы метода/алгоритма.
	%	\item Архитектура разработанной программы/библиотеки.
	%\end{enumerate}	
	%	\item Консультанты по работе\footnote{Подпись консультанта по нормоконтролю пока не требуется. Назначается всем по умолчанию.}:
	%	\begin{enumerate}[label=\theenumi\arabic*.] 
	%	\item  \uline{\emakefirstuc{\ConsultantExtraDegree}, \ConsultantExtra.} % закомментировать при необходимости, идёт первый по порядку.
	%	\item \uline{\emakefirstuc{\ConsultantNormDegree}, \ConsultantNorm{} (нормоконтроль).} %	Обязателен для всех студентов
	%\end{enumerate}
		\item Дата выдачи задания: \uline{\thesisStartDate.}
\end{enumerate}

\intervalS%можно удалить пробел

Руководитель ВКР \uline{\hspace*{0.3\textheight}} \Supervisor


%\intervalS%можно удалить пробел

%Консультант\footnote{В случае, если есть консультант, отличный от консультанта по нормоконтролю.}  \uline{\hspace*{0.1\textheight}\ConsultantExtra}


\intervalS%можно удалить пробел

%Консультант по нормоконтролю \uline{\hspace*{0.1\textheight} \ConsultantNorm}%ПОКА НЕ ТРЕБУЕТСЯ, Т.К. ОН У ВСЕХ ПО УМОЛЧАНИЮ

Задание принял к исполнению %\uline{\thesisStartDate}

\intervalS%можно удалить пробел

Студент \uline{\hspace*{0.4\textheight}} \Author



\setcounter{tskPageLast}{\value{page}} %сохранили номер последней страницы Задания
\setcounter{tskPages}{\value{tskPageLast}-\value{tskPageFirst}}
\newrefsection % начинаем новую секцию библиографии
\newrefcontext % удаляем префикс к пунктам списка литературы
\restoregeometry % восстанавливаем настройки страницы
\pagestyle{plain} % удаляем номер страницы на первой/второй странице Задания
\setlength{\parindent}{2.5em} % восстанавливаем абзацный отступ
%% Обязательно закомментировать, если получается один лист в задании:
\ifnumequal{\value{tskPages}}{0}{% Если 1 страница в Задании, то ничего не делать.
}{% Иначе 
% до 2020 года требовалось печатать задание на 1 листе с двух сторон и не подсчитывать вторую страницу
%\setcounter{page}{\value{page}-\value{tskPages}} 	% вычесть значение tskPages при печати более 1 страницы страниц
}%
\AtNextBibliography{\setcounter{citenum}{0}}%обнуляем счетчик библиографии	% настройки - конец