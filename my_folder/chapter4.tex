%\chapter{Название четвёртой главы. Апробация результатов исследования, а~именно: метода, алгоритма, модели исследования} \label{ch4}
%
%% не рекомендуется использовать отдельную section <<введение>> после лета 2020 года
%%\section{Введение} \label{ch4:intro}
%
%Хорошим стилем является наличие введения к главе. Во введении может быть описана цель написания главы, а также приведена краткая структура главы. 
%	
%\section{Название параграфа} \label{ch4:sec1}
%
%\section{Название параграфа} \label{ch4:sec2}
%
%Пример ссылки на литературу \cite{avtonomova:fya,Peskov2004-ru,Kotelnikov2004-ru,Kotelnikov2004}.
%
%%\FloatBarrier % заставить рисунки и другие подвижные (float) элементы остановиться
%
%\section{Выводы} \label{ch4:conclusion}
%
%Текст выводов по главе \thechapter.
%
%%% Вспомогательные команды - Additional commands
%%
%%\newpage % принудительное начало с новой страницы, использовать только в конце раздела
%%\clearpage % осуществляется пакетом <<placeins>> в пределах секций
%%\newpage\leavevmode\thispagestyle{empty}\newpage % 100 % начало новой страницы