\chapter*{Введение} % * не проставляет номер
\addcontentsline{toc}{chapter}{Введение} % вносим в содержание

В цифровой микроскопии применяются различные методы автофокусировки. Однако по-прежнему один из самых распространенных способов -- ручная фокусировка оператором микроскопа. Такой способ очевидно медленный и не самый точный. Большинство автоматических методов неустойчивы к шуму, что сильно влияет на конечный результат.

В последнее время разрабатываются все более качественные и универсальные подходы к автофокусировке микроскопов. Также нельзя не заметить активное использование нейросетей в широком спектре задач. Микроскопия не стала исключением. Но некоторые из методов автофокусировки них требуют внедрения в программное обеспечение (ПО) камеры.

Задачи, решаемые в рамках данной работы:
\begin{enumerate}
	\item Изучение существующих алгоритмов.
	\item Разработка нейросети, решающей задачу автофокусировки микроскопов.
	\item Проведение экспериментов и тестов.
\end{enumerate} 

Целью этой работы стала разработка нового нейросетевого метода автофокусировки, который бы удовлетворял следующим требованиям:
\begin{enumerate}
	\item Высокая скорость работы.
	\item Низкое потребление вычислительных ресурсов.
	\item Использование изображений только с камеры (без использования дополнительных датчиков).
\end{enumerate}


%% Вспомогательные команды - Additional commands
%\newpage % принудительное начало с новой страницы, использовать только в конце раздела
%\clearpage % осуществляется пакетом <<placeins>> в пределах секций
%\newpage\leavevmode\thispagestyle{empty}\newpage % 100 % начало новой строки