\chapter*{Заключение} \label{ch-conclusion}
\addcontentsline{toc}{chapter}{Заключение}	% в оглавление 

В данной работе были изучены существующие классические и нейросетевые методы автоматической фокусировки цифровых микроскопов. В ходе исследования было заключено, что нейросетевые подходы в этой области только начинают развиваться, однако имеют большой потенциал. Исходя из этого, была предложена нейронная сеть на основе глубокого обучения, решающая поставленную задачу. Резюмируя изложенный в главах 2 и 3 материал, можно заключить следующее:
\begin{itemize}
	\item Предложенный метод нейросетевого автофокуса за один шаг позволяет добиться лучшей точности, чем классические методы. С методами, использующими фазовые датчики, нейросеть можно сравнить лишь косвенно, однако, опираясь на материалы статьи, можно также сделать вывод, что фазовые методы уступают предложенному решению.
	
	\item Предложенное нейросетевое решение имеет более высокую скорость работы. Во многом это достигается именно благодаря отсутствию промежуточных перемещений камеры.
	
%	\item Алгоритм устойчив к шуму, возникающему во время съемки, что является еще одним его преимуществом.
	
	\item Предлагаемое решение можно развернуть на большинстве комплексов, так как данный метод не требует дополнительного оборудования, а большинство лабораторий применяет моторизованные микроскопы совместно с персональными компьютерами.
\end{itemize}

Стоит отметить, что для применения алгоритма в других областях может потребоваться дополнительное дообучение нейросети, поскольку нейронные сети при обучении выучивают основные признаки обучающего набора данных. Однако способность определять расстояние до фокальной плоскости остается, но точность в таком случае падает.