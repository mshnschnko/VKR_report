%%% Кодировки и шрифты %%%
\ifxetexorluatex
    \setmainlanguage[babelshorthands=true]{russian}  % Язык по-умолчанию русский с поддержкой приятных команд пакета babel
    \setotherlanguage{english}                       % Дополнительный язык = английский (в американской вариации по-умолчанию)
    \setmonofont{Courier New}
    \newfontfamily\cyrillicfonttt{Courier New}
    \ifXeTeX
        \defaultfontfeatures{Ligatures=TeX,Mapping=tex-text}
    \else
        \defaultfontfeatures{Ligatures=TeX}
    \fi
    \setmainfont{Times New Roman}
    \newfontfamily\cyrillicfont{Times New Roman}
    \setsansfont{Arial}
    \newfontfamily\cyrillicfontsf{Arial}
\else
    \IfFileExists{pscyr.sty}{\renewcommand{\rmdefault}{ftm}}{}
\fi

%%% Подписи %%%
\captionsetup{%
singlelinecheck=off,                % Многострочные подписи, например у таблиц
skip=5pt,                           % Вертикальная отбивка между подписью и содержимым рисунка или таблицы определяется ключом
justification=centering            % Центрирование подписей, заданных командой \caption
}

%\setlength{\abovecaptionskip}{0pt} %альтернатива для skip, но не распространяется на longtable!
%\setlength{\belowcaptionskip}{0pt}
%\captionwidth{\linewidth}
%\normalcaptionwidth

% для изменения отступов от floats (e.g. table,figure) & minipage
\newlength{\mfloatsep}
\setlength{\mfloatsep}{4mm plus 0.7mm minus 0.7mm} %3mm для A5

% фиксируем расстояния с помощью пакета layouts
\setlength{\textfloatsep}{\mfloatsep} % расстояние от текста до float, если float прижат к верхнему или нижнему краю
\setlength{\floatsep}{\mfloatsep} % расстояние от float до float (если оба сверху/снизу)
\setlength{\intextsep}{\mfloatsep} % расстояние от текста до float, если float снизу и сверху ограничен текстом 
%
%% фактически из-за бокса, внутрь которого помещается \captionof{figure} происходит увеличаение на 1мм отступа в соответствующем элементе!
%
%% по требованиям СПбПУ как раз необходим отступ 4мм от рисунка


%Возможно более гибко задавать отступы, например:
%\setlength{\floatsep}{12pt plus 2pt minus 2pt}
%\setlength{\textfloatsep}{20pt plus 2pt minus 4pt}
%\setlength{\intextsep}{\floatsep}

%https://tex.stackexchange.com/questions/60477/remove-space-after-figure-and-before-text
%https://tex.stackexchange.com/questions/26521/how-to-change-the-spacing-between-figures-tables-and-text




%%% Парный к \smallA шрифт 13bp в подписи%%%
%TO-DO как напрямую связать со \smallA
%\DeclareCaptionFont{font13bp}{\smallA\selectfont} %к сожалению, приводит к отсупу после номера рисунка
\DeclareCaptionFont{font13bp}{\fontsize{13bp}{16.77bp}\selectfont} %аналог задания вручную
\DeclareCaptionFont{font12bp}{\small\selectfont} %аналог задания вручную



%%% Рисунки %%%
\DeclareCaptionLabelSeparator*{emdash}{~--- }             % (ГОСТ 2.105, 4.3.1)

\DeclareCaptionLabelFormat{figwithoutspace}{#1#2}
%\captionsetup[figure]{labelformat=figwithoutspace,labelsep=none,name=Fig.}

\captionsetup[figure]{labelformat=figwithoutspace,labelsep=\figlabelsep,position=bottom,labelfont={font12bp},textfont={font12bp}}

%\setlength{\belowcaptionskip}{0pt} %расстояние между 
%\caption* -- подрисуночной подписи и
%\caption  -- подписи к рисунку с номером
%необходимо менять вслед за добавлением \vskip в \captionsetup

%\setfloatadjustment{figure}{%
%	\setlength{\belowcaptionskip}{-3pt}   % чтобы обивка после рисунков была 3mm, так как caption добавляет около 1мм к 3мм. 
%}




%%% Таблицы %%%
\ifnumequal{\value{tabcap}}{0}{%
    \newcommand{\tabcapalign}{\raggedright}  % по левому краю страницы или аналога parbox

    \DeclareCaptionFormat{tablecaption}{\tabcapalign #1#2#3}
    \captionsetup[table]{labelsep=emdash}        % тире как разделитель идентификатора с номером от наименования
}{%
    \ifnumequal{\value{tablaba}}{0}{%
        \newcommand{\tabcapalign}{\raggedright}  % по левому краю страницы или аналога parbox
    }{}

    \ifnumequal{\value{tablaba}}{1}{%
        \newcommand{\tabcapalign}{\centering}    % по центру страницы или аналога parbox
    }{}

    \ifnumequal{\value{tablaba}}{2}{%
        \newcommand{\tabcapalign}{\raggedleft}   % по правому краю страницы или аналога parbox
    }{}

    \ifnumequal{\value{tabtita}}{0}{%
        \newcommand{\tabtitalign}{\raggedright}  % по левому краю страницы или аналога parbox
    }{}

    \ifnumequal{\value{tabtita}}{1}{%
        \newcommand{\tabtitalign}{\centering}    % по центру страницы или аналога parbox
    }{}

    \ifnumequal{\value{tabtita}}{2}{%
        \newcommand{\tabtitalign}{\raggedleft}   % по правому краю страницы или аналога parbox
    }{}

    \DeclareCaptionFormat{tablecaption}{\tabcapalign #1#2\par %\hline  % Идентификатор таблицы на отдельной строке
        \tabtitalign{#3}}                                       % Наименование таблицы строкой ниже
    \captionsetup[table]{labelsep=\tablabelsep}                 % разделитель идентификатора с номером от наименования
}
\DeclareCaptionFormat{tablenocaption}{\tabcapalign #1\strut}    % Наименование таблицы отсутствует

\newlength{\ltskip}
\setlength{\ltskip}{2pt}
\captionsetup[table]{format=tablecaption,singlelinecheck=off,position=top,labelfont={font12bp},textfont={font12bp}}  % многострочные наименования и прочее
\DeclareCaptionLabelFormat{continued}{\\[-14pt]Продолжение табл.~\!#2}



%%% Подписи подрисунков %%%
\renewcommand{\thesubfigure}{\alph{subfigure}}           % Буквенные номера подрисунков
\captionsetup[subfigure]{font={font12bp},               % Шрифт подписи названий подрисунков (отличается от основного)
	labelfont={font12bp},textfont={font12bp},
    labelformat=brace,                                    % Формат обозначения подрисунка
    singlelinecheck=off,
%    position=left,
    justification=raggedright 							 %выравнивание влево
%    justification=centering,                              % Выключка подписей (форматирование), один из вариантов            
}

%%% Подписи подрисунков SPbPU%%%
% реализован подход по первой ссылке, он позволяет масштабировать количество подрисунков
%https://tex.stackexchange.com/a/273169/44348
%https://tex.stackexchange.com/a/225914/44348
\usepackage[export]{adjustbox}



%%% Настройки гиперссылок %%%
\ifLuaTeX
    \hypersetup{
        unicode,                % Unicode encoded PDF strings
    }
\fi


\newcommand{\thesisTitle}{Название выпускной квалификационной работы}


\hypersetup{
    linktocpage=true,           % ссылки с номера страницы в оглавлении, списке таблиц и списке рисунков
%    linktoc=all,                % both the section and page part are links
%    pdfpagelabels=false,        % set PDF page labels (true|false)
    plainpages=false,           % Forces page anchors to be named by the Arabic form  of the page number, rather than the formatted form
    %colorlinks,                 % ссылки отображаются раскрашенным текстом, а не раскрашенным прямоугольником, вокруг текста
    citebordercolor={0.287 0.89 0.349}, %(RGB colour) with default {0 1 0}: The colour of the box around citations
    filebordercolor={0 .5 .5}, % (RGB colour) with default {0 .5 .5}: The colour of the box around links to files
    linkbordercolor={0.93 0 0}, % (RGB colour) with default {1 0 0}: The colour of the box around normal links
    menubordercolor={1 0 0}, % (RGB colour) with default {1 0 0}: The colour of the box around Acrobat menu links
    urlbordercolor={0.313 0.776 0.878}, % (RGB colour) with default {0 1 1}: The colour of the box around links to URLs
    pdfborder={0 0 1}, % The style of box around links; defaults to a box with lines of 1pt thickness, but the colorlinks option resets it to produce no border.
%    linkcolor={linkcolor},
%    citecolor={citecolor},      % цвет ссылок-цитат
%    urlcolor={urlcolor},        % цвет гиперссылок
    %hidelinks,                  % Hide links (removing color and border)
%    pdftitle={\thesisTitle},    % Заголовок pdf-файла
%    pdfauthor={\AuthorFull},  % Автор
%    pdfsubject={\thesisSpecialtyNumber\ \thesisSpecialtyTitle},      % Тема
%    pdfcreator={Создатель},     % Создатель, Приложение
%    pdfproducer={Производитель},% Производитель, Производитель PDF
%    pdfkeywords={\keywords},    % Ключевые слова
    pdflang={eng}, %eng %ru
    % % bookmarks settings
    bookmarks=true,
    bookmarksnumbered=true, % put section numbers
    bookmarksopen=true, %open when the pdf is opened
    bookmarksopenlevel=0, %chapter's level is enough to see
    bookmarksdepth=0 %set the depth of the levels in the pdf navigation bar
}

% %improve the bookmarksnumbered representation:
\makeatletter
\renewcommand{\Hy@numberline}[1]{#1. } %add the dot after a number
\makeatother


\ifnumequal{\value{draft}}{1}{% Черновик
    \hypersetup{
        draft,
    }
}{}

%%% Шаблон %%%
\DeclareRobustCommand{\todo}{\textcolor{red}}       % решаем проблему превращения названия цвета в результате \MakeUppercase, http://tex.stackexchange.com/a/187930/79756 , \DeclareRobustCommand protects \todo from expanding inside \MakeUppercase
\AtBeginDocument{%
    \setlength{\parindent}{2.5em}                   % Абзацный отступ. Должен быть одинаковым по всему тексту и равен пяти знакам (ГОСТ Р 7.0.11-2011, 5.3.7).
}

%%% Списки %%%
% Используем короткое тире (endash) для ненумерованных списков (ГОСТ 2.105-95, пункт 4.1.7, требует дефиса, но так лучше смотрится)
%\renewcommand{\labelitemi}{\normalfont\bfseries{--}}

%% Перечисление строчными буквами латинского алфавита (ГОСТ 2.105-95, 4.1.7)
\renewcommand{\theenumi}{\Alph{enumi}} % первый уровень иерархии %латинскийалфавит заглавные
\renewcommand{\labelenumi}{\theenumi.} 
%\renewcommand{\theenumii}{\alph{enumii}} % второй уровень иерархии %латинскийалфавит
%\renewcommand{\labelenumii}{\theenumii)} 
%
%
%% Перечисление строчными буквами русского алфавита (ГОСТ 2.105-95, 4.1.7)
\makeatletter
\AddEnumerateCounter{\asbuk}{\russian@alph}{щ}      % Управляем списками/перечислениями через пакет enumitem, а он 'не знает' про asbuk, потому 'учим' его
\makeatother
%%\renewcommand{\theenumi}{\asbuk{enumi}} %первый уровень нумерации
%%\renewcommand{\labelenumi}{\theenumi)} %первый уровень нумерации 
%%\renewcommand{\theenumii}{\asbuk{enumii}} %второй уровень нумерации
%%\renewcommand{\labelenumii}{\theenumii)} %второй уровень нумерации 
\renewcommand{\theenumii}{\arabic{enumii}} %второй уровень нумерации %арабские цифры
\renewcommand{\labelenumii}{\theenumii.} %второй уровень нумерации 
%\renewcommand{\theenumi}{\arabic{enumi}} %первый уровень нумерации %арабские цифры
%\renewcommand{\labelenumi}{\theenumi)} %первый уровень нумерации 
%
%\renewcommand{\theenumiii}{\asbuk{enumiii}} %третий уровень нумерации %русский алфавит
\renewcommand{\theenumiii}{\alph{enumiii}} %третий уровень нумерации %английский алфавит
\renewcommand{\labelenumiii}{\theenumiii)} %третий уровень нумерации 
%\renewcommand{\theenumiii}{\arabic{enumiii}} %третий уровень нумерации %арабские цифры
%\renewcommand{\labelenumiii}{\theenumiii)} %третий уровень нумерации 



\setlist{nosep,%                                    % Единый стиль для всех списков (пакет enumitem), без дополнительных интервалов.
    labelindent=\parindent,leftmargin=*%            % Каждый пункт, подпункт и перечисление записывают с абзацного отступа (ГОСТ 2.105-95, 4.1.8)
}
%\setlist[enumerate]{label*=\arabic*.}



% asm packages required! In particular amsthm
%http://tex.stackexchange.com/questions/37472/spacing-before-and-after-with-newtheoremstyle

%theoremstyle{}
%plain : italic text, extra space above and below;
%definition : upright text, extra space above and below;
%remark : upright text, no extra space above or below.

\newtheoremstyle{myplain} %
{0} %space above
{0} %space below
{\itshape}
{\parindent}
{\bfseries}
{.}
{.5em}
{}

\newtheoremstyle{mydefinition} %
{0} %space above
{0} %space below
{}
{\parindent}
{\bfseries}
{.}
{.5em}
{}

\theoremstyle{myplain} %improved plain style
\newtheorem{m-theorem}{Теорема}[chapter] % reset theorem numbering for each chapter
\newtheorem{m-corollary}{Следствие}[chapter] % definition numbers are 
\newtheorem{m-proposition}{Утверждение}[chapter] % definition numbers are dependent on theorem numbers
\newtheorem{m-lemma}{Лемма}[chapter]
\newtheorem{m-axiom}{Аксиома}[chapter]

\theoremstyle{mydefinition} %improved definition style
\newtheorem{m-example}{Пример}[chapter] % same for example numbers
\newtheorem{m-definition}{Определение}[chapter]  % definition numbers
\newtheorem{m-condition}{Условие}[chapter]
\newtheorem{m-problem}{Проблема}[chapter]
\newtheorem{m-exercise}{Упражнение}[chapter]
\newtheorem{m-question}{Вопрос}[chapter]
\newtheorem{m-hypothesis}{Гипотеза}[chapter]
\newtheorem{m-task}{Задание}[chapter]



%%control skip of thm, plain style - ANOTHER VARIANT
%%http://tex.stackexchange.com/questions/85400/how-to-change-space-around-theorem-environments
%\makeatletter
%\def\thm@space@setup{%
%	\thm@preskip=0cm %
%	%	\thm@preskip=0cm plus 0.2cm minus 0.2cm
%	\thm@postskip=0cm % or whatever, if you don't want them to be equal
%	%	\thm@postskip=\thm@preskip % or whatever, if you don't want them to be equal
%}
%\makeatother
