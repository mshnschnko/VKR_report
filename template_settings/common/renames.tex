%%% Внесите свои данные - Input your data
%%
%%
\newcommand{\Author}{М.Д.\,Анищенко} % И.О. Фамилия автора 
\newcommand{\AuthorFull}{Анищенко Михаил Денисович} % Фамилия Имя Отчество автора
\newcommand{\AuthorFullDat}{Анищенко Михаилу Денисовичу} % Фамилия Имя Отчество автора в дательном падеже (Кому? Студенту...)
\newcommand{\AuthorFullVin}{Анищенко Михаила Денисовича} % в винительном падеже (Кого? что?  Програмиста ...)
\newcommand{\AuthorPhone}{+7-987-642-29-86} % номер телефорна автора для оперативной связи  
\newcommand{\Supervisor}{В.С.\,Чуканов} % И. О. Фамилия научного руководителя
\newcommand{\SupervisorFull}{Чуканов Вячеслав Сергеевич} % Фамилия Имя Отчество научного руководителя
\newcommand{\SupervisorVin}{В.С.\,Чуканова} % И. О. Фамилия научного руководителя  в винительном падеже (Кого? что? Руководителя ...)
\newcommand{\SupervisorJob}{старший преподаватель ВШПМиВФ} %
\newcommand{\SupervisorJobVin}{старшего преподавателя ВШПМиВФ} % в винительном падеже (Кого? что?  Програмиста ...)
\newcommand{\SupervisorDegree}{} %
\newcommand{\SupervisorTitle}{} % 
%%
%%
%Руководитель, утверждающий задание
\newcommand{\Head}{К.Н.\,Козлов} % И. О. Фамилия руководителя подразделения (руководителя ОП)
\newcommand{\HeadDegree}{Руководитель образовательной\\&программы}% Только должность:   
%Руководитель %ОП 
%Заведующий % кафедрой
%Директор % Высшей школы
%Зам. директора
\newcommand{\HeadDep}{<<Прикладная математика\\&и информатика>>} % заменить на краткую аббревиатуру подразделения или оставить пустым, если утверждает руководитель ОП

%%% Руководитель, принимающий заявление
\newcommand{\HeadAp}{К.Н.\,Козлов} % И. О. Фамилия руководителя подразделения (руководителя ОП)
\newcommand{\HeadApDegree}{Руководитель образовательной\\ программы}% Только должность:   
%Руководитель ОП 
%Заведующий кафедрой
%Директор Высшей школы
\newcommand{\HeadApDep}{<<Прикладная математика\\ и информатика>>} % заменить на краткую аббревиатуру подразделения или оставить пустым, если утверждает руководитель ОП
%%% Консультант по нормоконтролю
\newcommand{\ConsultantNorm}{Л.А.\,Арефьева} % И. О. Фамилия консультанта по нормоконтролю. ТОЛЬКО из числа ППС!
\newcommand{\ConsultantNormDegree}{должность, степень} %   
%%% Первый консультант
\newcommand{\ConsultantExtraFull}{Фамилия Имя Отчетство} % Фамилия Имя Отчетство дополнительного консультанта 
\newcommand{\ConsultantExtra}{И.О.\,Фамилия} % И. О. Фамилия дополнительного консультанта 
\newcommand{\ConsultantExtraDegree}{должность, степень} % 
\newcommand{\ConsultantExtraVin}{И.О.\,Фамилию} % И. О. Фамилия дополнительного консультанта в винительном падеже (Кого? что? Руководителя ...)
\newcommand{\ConsultantExtraDegreeVin}{должность, степень} %  в винительном падеже (Кого? что? Руководителя ...)
%%% Второй консультант
\newcommand{\ConsultantExtraTwoFull}{Фамилия Имя Отчетство} % Фамилия Имя Отчетство дополнительного консультанта 
\newcommand{\ConsultantExtraTwo}{И.О.\,Фамилия} % И. О. Фамилия дополнительного консультанта 
\newcommand{\ConsultantExtraTwoDegree}{должность, степень} % 
\newcommand{\ConsultantExtraTwoVin}{И.О.\,Фамилию} % И. О. Фамилия дополнительного консультанта в винительном падеже (Кого? что? Руководителя ...)
\newcommand{\ConsultantExtraTwoDegreeVin}{должность, степень} %  в винительном падеже (Кого? что? Руководителя ...)
\newcommand{\Reviewer}{И.О.\,Фамилия} % И. О. Фамилия резензента. Обязателен только для магистров.
\newcommand{\ReviewerDegree}{должность, степень} % 
%%
%%
\renewcommand{\thesisTitle}{Нейросетевые и классические методы для задачи автофокуса цифрового микроскопа}
\newcommand{\thesisDegree}{работа бакалавра}% дипломный проект, дипломная работа, магистерская диссертация %c 2020
\newcommand{\thesisTitleEn}{Title of the thesis} %2020
\newcommand{\thesisDeadline}{июнь 2024}
\newcommand{\thesisStartDate}{25.01.2024}
\newcommand{\thesisYear}{2024}
%%
%%
\newcommand{\group}{5030102/00201} % заменить вместо N номер группы
\newcommand{\thesisSpecialtyCode}{01.03.02}% код направления подготовки
\newcommand{\thesisSpecialtyTitle}{Прикладная математика и информатика} % наименование направления/специальности
\newcommand{\thesisOPPostfix}{02} % последние цифры кода образовательной программы (после <<_>>)
\newcommand{\thesisOPTitle}{Системное программирование}% наименование образовательной программы
%%
%%
\newcommand{\institute}{
Физико-механический институт
%Название института
%Институт компьютерных наук и~технологий
%Гуманитарный институт
%Инженерно-строительный институт
%Институт биомедицинских систем и технологий
%Институт металлургии, машиностроения и транспорта
%Институт передовых производственных технологий
%Институт прикладной математики и механики
%Институт физики, нанотехнологий и телекоммуникаций
%Институт физической культуры, спорта и туризма
%Институт энергетики и транспортных систем
%Институт промышленного менеджмента, экономики и торговли
}%
%%
%%




%%% Задание ключевых слов и аннотации
%%
%%
%% Ключевых слов от 3 до 5 слов или словосочетаний в именительном падеже именительном падеже множественного числа (или в единственном числе, если нет другой формы) по правилам русского языка!!!
%%
%%
\newcommand{\keywordsRu}{Автофокусировка, Цифровая микроскопия, Свёрточные нейронные сети, Архитектура сети, Фоклаьная плоскость} % ВВЕДИТЕ ключевые слова по-русски
%%
%%
\newcommand{\keywordsEn}{Autofocus, Digital microscopy, Convolutional neural networks, Network architecture, Focal plane} % ВВЕДИТЕ ключевые слова по-английски
%%
%%
%% Реферат ОТ 1000 ДО 1500 знаков на русский или английский текст
%%
%Реферат должен содержать:
%- предмет, тему, цель ВКР;
%- метод или методологию проведения ВКР:
%- результаты ВКР:
%- область применения результатов ВКР;
%- выводы.
\newcommand\tab[1][1.3cm]{\hspace*{#1}}
\newcommand{\abstractRu}{Данная работа посвящена различным методом автофокусировки цифровых микроскопов, а также разработке и реализации нового подхода к этой задаче с применением нейронных сетей и глубокого обучения.\\\tab В результате был разработан алгоритм автофокусировки цифрового микроскопа, который не зависит от конкретного оборудования и не требует интеграции в программное обеспечение микроскопа или камеры. Алгоритм основан на архитектуре сверточной нейронной сети MobileNetV3. Сеть обучалась на наборе данных из открытого источника, состоящем из снимков с микроскопа, сделанных при 20-кратном увеличении, и целых чисел, обозначающих расстояние до фокальной плоскости с учетом направления.\\\tab По результатам тестирования был сделан следующий вывод: представленный алгоритм можно активно применять на практике, так как он имеет высокую скорость работы и отличное качество, не уступающее классическим методам, а также более устойчив к шуму и прост во внедрении.} % ВВЕДИТЕ текст аннотации по-русски
%%
%%
\newcommand{\abstractEn}{This work is devoted to various methods of autofocus of digital microscopes, as well as the development and implementation of a new approach to this task using neural networks and deep learning. \\\tab As a result, a digital microscope autofocus algorithm has been developed that does not depend on specific hardware and does not require integration into the microscope or camera software. The algorithm is based on the architecture of the convolutional neural network Mobile Net V3. The network was trained on an open source dataset consisting of microscope images taken at 20x magnification and integers indicating the distance to the focal plane, taking into account the direction. \\\tab Based on the test results, the following conclusion was made: the presented algorithm can be actively applied in practice, since it has a high speed and excellent quality, not inferior to classical methods, as well as more resistant to noise and easy to implement.} % ВВЕДИТЕ текст аннотации по-английски


%%% РАЗДЕЛ ДЛЯ ОФОРМЛЕНИЯ ПРАКТИКИ
%Место прохождения практики
\newcommand{\PracticeType}{Отчет о прохождении % 
	%стационарной производственной (технологической (проектно-технологической)) %
	преддипломной % тип и вид ЗАМЕНИТЬ
	практики}

\newcommand{\Workplace}{СПбПУ, ФизМех, ВШПМиВФ} % TODO Rename this variable

% Даты начала/окончания
\newcommand{\PracticeStartDate}{%
03.05.2024%
%	22.06.2020
}%
\newcommand{\PracticeEndDate}{%
	31.05.2024%
%	18.07.2020%
}%
%%

\newcommand{\School}{
	Высшая школа прикладной математики и вычислительной физики
%	Высшая школа интеллектуальных систем и~суперкомпьютерных~технологий 
}
\newcommand{\practiceTitle}{Нейросетевые и классические методы для задачи автофокуса цифрового микроскопа}


%% ВНИМАНИЕ! Необходимо либо заменить текст аннотации (ключевых слов) на русском и английском, либо удалить там весь текст, иначе в свойства pdf-отчета по практике пойдет шаблонный текст.

%%% Не меняем дальнейшую часть - Do not modify the rest part
%%
%%
%%
%%
\ifnumequal{\value{docType}}{1}{% Если ВКР, то...
	\newcommand{\DocType}{Выпускная квалификационная работа}
	\newcommand{\pdfDocType}{\DocType~(\thesisDegree)} %задаём метаданные pdf файла
	\newcommand{\pdfTitle}{\thesisTitle}
}{% Иначе 
	\newcommand{\DocType}{\PracticeType}
	\newcommand{\pdfDocType}{\DocType} %задаём метаданные pdf файла
	\newcommand{\pdfTitle}{\practiceTitle}
}%
\newcommand{\HeadTitle}{\HeadDegree~\HeadDep}
\newcommand{\HeadApTitle}{\HeadApDegree~\HeadApDep}
\newcommand{\thesisOPCode}{\thesisSpecialtyCode\_\thesisOPPostfix}% код образовательной программы
\newcommand{\thesisSpecialtyCodeAndTitle}{\thesisSpecialtyCode~\thesisSpecialtyTitle}% Код и наименование направления/специальности
\newcommand{\thesisOPCodeAndTitle}{\thesisOPCode~\thesisOPTitle} % код и наименование образовательной программы
%%
%%
\hypersetup{%часть болка hypesetup в style
		pdftitle={\pdfTitle},    % Заголовок pdf-файла
		pdfauthor={\AuthorFull},    % Автор
		pdfsubject={\pdfDocType. Шифр и наименование направления подготовки: \thesisSpecialtyCodeAndTitle. \abstractRu},      % Тема
		pdfcreator={LaTeX, SPbPU-student-thesis-template},     % Приложение-создатель
%		pdfproducer={},  % Производитель, Производитель PDF % будет выставлена автоматически
		pdfkeywords={\keywordsRu}
}
%%
%%
%% вспомогательные команды
\newcommand{\firef}[1]{рис.\ref{#1}} %figure reference
\newcommand{\taref}[1]{табл.\ref{#1}}	%table reference
%%
%%
%% Архивный вариант задания ключевых слов, аннотации и благодарностей 
% Too hard to export data from the environment to pdf-info
% https://tex.stackexchange.com/questions/184503/collecting-contents-of-environment-and-store-them-for-later-retrieval
%заменить NewEnviron на newenvironment для распознавания команды в TexStudio
%\NewEnviron{keywordsRu}{\noindent\MakeUppercase{\BODY}}
%\NewEnviron{keywordsEn}{\noindent\MakeUppercase{\BODY}}
%\newenvironment{abstractRu}{}{}
%\newenvironment{abstractEn}{}{}
%\newenvironment{acknowledgementsRu}{\par{\normalfont \acknowledgements.}}{}
%\newenvironment{acknowledgementsEn}{\par{\normalfont \acknowledgementsENG.}}{}


%%% Переопределение именований %%% Не меняем - Do not modify
%\newcommand{\Ministry}{Минобрнауки России} 
\newcommand{\Ministry}{Министерство науки и высшего образования Российской~Федерации} %с 2020
\newcommand{\SPbPU}{Санкт-Петербургский политехнический университет Петра~Великого}
\newcommand{\SPbPUOfficialPrefix}{Федеральное государственное автономное образовательное учреждение высшего образования}
\newcommand{\SPbPUOfficialShort}{ФГАОУ~ВО~<<СПбПУ>>}
%% Пробел между И. О. не допускается.
\renewcommand{\alsoname}{см. также}
\renewcommand{\seename}{см.}
\renewcommand{\headtoname}{вх.}
\renewcommand{\ccname}{исх.}
\renewcommand{\enclname}{вкл.}
\renewcommand{\pagename}{Pages}
\renewcommand{\partname}{Часть}
\renewcommand{\abstractname}{\textbf{Аннотация}}
\newcommand{\abstractnameENG}{\textbf{Annotation}}
\newcommand{\keywords}{\textbf{Ключевые слова}}
\newcommand{\keywordsENG}{\textbf{Keywords}}
\newcommand{\acknowledgements}{\textbf{Благодарности}}
\newcommand{\acknowledgementsENG}{\textbf{Acknowledgements}}
\renewcommand{\contentsname}{Content} % 
%\renewcommand{\contentsname}{Содержание} % (ГОСТ Р 7.0.11-2011, 4)
%\renewcommand{\contentsname}{Оглавление} % (ГОСТ Р 7.0.11-2011, 4)
\renewcommand{\figurename}{Рис.} % Стиль СПбПУ
%\renewcommand{\figurename}{Рисунок} % (ГОСТ Р 7.0.11-2011, 5.3.9)
\renewcommand{\tablename}{Таблица} % (ГОСТ Р 7.0.11-2011, 5.3.10)
%\renewcommand{\indexname}{Предметный указатель}
\renewcommand{\listfigurename}{Список рисунков}
\renewcommand{\listtablename}{Список таблиц}
\renewcommand{\refname}{\fullbibtitle}
\renewcommand{\bibname}{\fullbibtitle}

\newcommand{\chapterEnTitle}{Сhapter title} % <- input the English title here (only once!) 
\newcommand{\chapterRuTitle}{Название главы}          % <- введите 
\newcommand{\sectionEnTitle}{Section title} %<- input subparagraph title in english
\newcommand{\sectionRuTitle}{Название подраздела} % <- введите название подраздела по-русски
\newcommand{\subsectionEnTitle}{Subsection title} % - input subsection title in english
\newcommand{\subsectionRuTitle}{Название параграфа} % <- введите название параграфа по-русски
\newcommand{\subsubsectionEnTitle}{Subsubsection title} % <- input subparagraph title in english
\newcommand{\subsubsectionRuTitle}{Название подпараграфа} % <- введите название подпараграфа по-русски